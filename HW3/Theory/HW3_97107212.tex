\documentclass[12pt]{article}
\usepackage{graphicx,import}
\usepackage[svgnames]{xcolor} 
\usepackage{fancyhdr}
\usepackage{subfig}
\usepackage{hyperref}
\usepackage{enumitem}
\usepackage{cite}
\usepackage[many]{tcolorbox}
\usepackage{listings }
\usepackage[a4paper, total={6in, 8in} , bottom = 25mm , top = 25mm, headheight = 1.25cm , includehead,includefoot,heightrounded ]{geometry}
\usepackage{afterpage}
\usepackage{amssymb}
\usepackage{pdflscape}
\usepackage{gensymb}
\usepackage{textcomp}
\usepackage{tikz,pgfplots}
\usepackage{xecolor}
\usepackage{rotating}
\usepackage{pdfpages}
\usepackage{fancyvrb}
\usepackage[Kashida]{xepersian}
\usepackage[T1]{fontenc}
\usepackage{tikz}
\usepackage[utf8]{inputenc}
\usepackage{PTSerif} 
\usepackage{seqsplit}

\usepackage[edges]{forest}

\usepackage{listings}
\usepackage{xcolor}

\hypersetup{
	colorlinks   = true, %Colours links instead of ugly boxes
	urlcolor     = blue, %Colour for external hyperlinks
	linkcolor    = blue, %Colour of internal links
	citecolor   = red %Colour of citations
}

\definecolor{codegreen}{rgb}{0,0.6,0}
\definecolor{codegray}{rgb}{0.5,0.5,0.5}
\definecolor{codepurple}{rgb}{0.58,0,0.82}
\definecolor{backcolour}{rgb}{0.95,0.95,0.92}

\NewDocumentCommand{\codeword}{v}{
	\texttt{\textcolor{blue}{#1}}
}
\lstset{language=java,keywordstyle={\bfseries \color{blue}}}

\lstdefinestyle{mystyle}{
	backgroundcolor=\color{backcolour},   
	commentstyle=\color{codegreen},
	keywordstyle=\color{magenta},
	numberstyle=\tiny\color{codegray},
	stringstyle=\color{codepurple},
	basicstyle=\ttfamily\normalsize,
	breakatwhitespace=false,         
	breaklines=true,                 
	captionpos=b,                    
	keepspaces=true,                 
	numbers=left,                    
	numbersep=5pt,                  
	showspaces=false,                
	showstringspaces=false,
	showtabs=false,                  
	tabsize=2
}

\lstset{style=mystyle}

\settextfont[Scale=1.2 ,BoldFont={Bahij Nazanin-Bold.ttf} , ItalicFont = {IRNazaninIranic.ttf}]{Bahij Nazanin-Regular.ttf}
\setlatintextfont[Scale = 1.0]{Garamond}
\DefaultMathsDigits 
\DeclareMathSizes{11}{19}{13}{9} 
%\DeclareMathSizes{12}{14.4}{8}{9}





\newenvironment{changemargin}[2]{%
	\begin{list}{}{%
			\setlength{\topsep}{0pt}%
			\setlength{\leftmargin}{#1}%
			\setlength{\rightmargin}{#2}%
			\setlength{\listparindent}{\parindent}%
			\setlength{\itemindent}{\parindent}%
			\setlength{\parsep}{\parskip}%
		}%
		\item[]}{\end{list}}


\definecolor{foldercolor}{RGB}{124,166,198}

\tikzset{pics/folder/.style={code={%
			\node[inner sep=0pt, minimum size=#1](-foldericon){};
			\node[folder style, inner sep=0pt, minimum width=0.3*#1, minimum height=0.6*#1, above right, xshift=0.05*#1] at (-foldericon.west){};
			\node[folder style, inner sep=0pt, minimum size=#1] at (-foldericon.center){};}
	},
	pics/folder/.default={20pt},
	folder style/.style={draw=foldercolor!80!black,top color=foldercolor!40,bottom color=foldercolor}
}

\forestset{is file/.style={edge path'/.expanded={%
			([xshift=\forestregister{folder indent}]!u.parent anchor) |- (.child anchor)},
		inner sep=1pt},
	this folder size/.style={edge path'/.expanded={%
			([xshift=\forestregister{folder indent}]!u.parent anchor) |- (.child anchor) pic[solid]{folder=#1}}, inner xsep=0.6*#1},
	folder tree indent/.style={before computing xy={l=#1}},
	folder icons/.style={folder, this folder size=#1, folder tree indent=3*#1},
	folder icons/.default={12pt},
}

\begin{document}
	
	
	%%% title pages
	\begin{titlepage}
		\begin{center}
			
			\vspace*{0.7cm}
			
			\includegraphics[width=0.4\textwidth]{sharif1.png}\\
			\vspace{0.5cm}
			\textbf{ \Huge{\emph  ﺷﺒﻜﻪ‌های کامپیوتری} }\\
			\vspace{0.5cm}
			\textbf{ \Large{ تمرین سوم} }
			\vspace{0.2cm}
			
			
			\large \textbf{دانشکده مهندسی کامپیوتر}\\\vspace{0.2cm}
			\large   دانشگاه صنعتی شریف\\\vspace{0.2cm}
			\large   ﻧﯿﻢ سال دوم 00-99 \\\vspace{0.2cm}
			\noindent\rule[1ex]{\linewidth}{1pt}
			استاد:\\
			\textbf{{جناب آقای دکتر جعفری}}
			
			
			\vspace{0.15cm}
			نام و نام خانوادگی:\\
			
			
			\textbf{{امیرمهدی نامجو - 97107212}}
		\end{center}
	\end{titlepage}
	%%% title pages
	
	
	%%% header of pages
	\newpage
	\pagestyle{fancy}
	\fancyhf{}
	\fancyfoot{}
	\cfoot{\thepage}
	\chead{تمرین سوم}
	\rhead{\includegraphics[width=0.1\textwidth]{sharif.png}}
	\lhead{امیرمهدی نامجو}
	%%% header of pages
	
	\KashidaOff
	
	\section{سوال اول}


روش \lr{UDP Hole Punching} روشی است که به کمک آن می‌توان ارتباط بین دو کلاینت که یک یا هر دوی آن ها پشت NAT قرار دارند را برقرار کرد. در اصل این روش به نوعی یک حفره در دیواره NAT ایجاد می‌کند و برای همین \lr{Hole Punching} نام دارد. نحوه کار این روش بدین صورت است:

فرض کنید می‌خواهیم ارتباط بین A و B را برقار کنیم. در روش \lr{Hole Punching} نیاز به داشتن یک واسطه مانند C است که هر دوی A و B آدرس IP آن را بداند.

در مرحله اول \lr{A} و \lr{B} هر دو پکت‌های \lr{UDP} را به \lr{C} می‌فرستند. با عبور پکت های آنان از \lr{NAT} شان، این \lr{NAT}، آدرس \lr{IP} مبدا این پکت‌ها را بازنویسی می‌کند تا مشخص باشد که پاسخ آن باید به کجا ارسال شود.

در مرحله دوم، \lr{C} متوجه \lr{IP} آدرس و همچنین پورت درخواست هایی که از سمت \lr{A} و \lr{B} آمده‌اند می‌شود. (مثلا فرض کنید پورت \lr{A} برابر \lr{X} و پورت \lr{B} برابر \lr{Y }باشد) با توجه به ساختار عمومی \lr{NAT}، در حال حاضر \lr{C }می‌تواند به راحتی از این طریق با \lr{A} و \lr{B} ارتباط برقرار کند و با ارسال پیام به \lr{NAT} هر کدام از آن‌ها، از آن جایی که \lr{NAT} می‌داند که شروع درخواست از سمت قسمت‌های درونی خود بوده است و اطلاعات را دارد، بسته را به درستی به مقصد می‌رساند.

در مرحله بعد، \lr{C} به \lr{A} پیامی می‌دهد که می گوید برای ارتباط برقرار کردن با \lr{B}، برای آدرس \lr{IP }مربوط به \lr{NAT} آن و پورت \lr{Y} پیام ارسال کن. از طرفی به \lr{B }هم می‌گوید برای ارتباط برقرار کردن با \lr{A} به آدرس \lr{IP} مربوط به \lr{NAT} آن و پورت \lr{X} پیام ارسال کن.

در مرحله بعد، ابتدا اولین پکت های ارسالی از سمت \lr{A} و \lr{B} به درستی به مقصد نمی رسد و توسط \lr{NAT }های مربوطه \lr{Reject} می‌شود. اما با ارسال اولین پیام از سمت \lr{A} به \lr{B} و عبور آن از \lr{NAT} مربوط به \lr{A}، این \lr{NAT }متوجه می‌شود که \lr{A} قصد ارتباط برقرار کردن با \lr{IP} آدرسی که مربوط به \lr{NAT} هاست \lr{B} است و پورت \lr{Y} را دارد و از این رو پیام های دریافتی بعدی از این آردس را برای \lr{A} می فرستد. همین اتفاق از سمت \lr{NAT} دیگر هم می افتد. از این به بعد این دو \lr{NAT} می‌دانند درخواست هایی که از سمت مقابل می‌آید را باید به کدام یک از \lr{Host} های سمت خود تحویل بدهند. به نوعی یک حفره در \lr{NAT} ایجاد شده که درخواست‌هایی که از آدرس خاصی می‌آیند را به درستی به یکدیگر تحویل می‌دهد. بدین ترتیب ارتباط \lr{P2P} بین \lr{A} و \lr{B} برقرار می‌شود.


\end{document}



